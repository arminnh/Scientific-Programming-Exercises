\documentclass[11pt, a4paper, titlepage, openright]{article}
% \usepackage[options]{package}

\title{\LARGE Scientific Programming \\ \normalsize Data Fitting Exercise 1}
\author{Armin Halilovic - s0122210}
\date{October 30, 2015}

\usepackage[font=small,labelfont=bf]{caption}
\usepackage{float}
%\floatstyle{boxed}
\restylefloat{figure}
\usepackage{graphicx}
\usepackage{hyperref}
\usepackage{mathtools}
%\usepackage{titlesec}
\usepackage[titletoc, title]{appendix}
\usepackage{listings}
\usepackage{color}
\usepackage[a4paper, total={6.5in, 9.5in}]{geometry}

\definecolor{dkgreen}{rgb}{0,0.6,0}
\definecolor{gray}{rgb}{0.5,0.5,0.5}
\definecolor{mauve}{rgb}{0.58,0,0.82}

\lstset{frame=tb,
  language=C++,
  aboveskip=3mm,
  belowskip=3mm,
  showstringspaces=false,
  columns=flexible,
  basicstyle={\footnotesize\ttfamily},
  numbers=none,
  numberstyle=\tiny\color{gray},
  keywordstyle=\color{blue},
  commentstyle=\color{dkgreen},
  stringstyle=\color{mauve},
  breaklines=true,
  breakatwhitespace=true,
  tabsize=3,
  showstringspaces=false
}

\begin{document}
%%%%%%%%%%%%%%%%%%%%%%%%%%%%%%%%%%%%%%%%%
% University Assignment Title Page
% LaTeX Template
% Version 1.0 (27/12/12)
%
% This template has been downloaded from:
% http://www.LaTeXTemplates.com
%
% Original author:
% WikiBooks (http://en.wikibooks.org/wiki/LaTeX/Title_Creation)
%
% License:
% CC BY-NC-SA 3.0 (http://creativecommons.org/licenses/by-nc-sa/3.0/)
%
% Instructions for using this template:
% This title page is capable of being compiled as is. This is not useful for
% including it in another document. To do this, you have two options:
%
% 1) Copy/paste everything between \begin{document} and \end{document}
% starting at \begin{titlepage} and paste this into another LaTeX file where you
% want your title page.
% OR
% 2) Remove everything outside the \begin{titlepage} and \end{titlepage} and
% move this file to the same directory as the LaTeX file you wish to add it to.
% Then add \input{./title_page_1.tex} to your LaTeX file where you want your
% title page.
%
%%%%%%%%%%%%%%%%%%%%%%%%%%%%%%%%%%%%%%%%%

%----------------------------------------------------------------------------------------
%	PACKAGES AND OTHER DOCUMENT CONFIGURATIONS
%----------------------------------------------------------------------------------------
\begin{titlepage}

\newcommand{\HRule}{\rule{\linewidth}{0.5mm}} % Defines a new command for the horizontal lines, change thickness here

\center % Center everything on the page

%----------------------------------------------------------------------------------------
%	HEADING SECTIONS
%----------------------------------------------------------------------------------------

\textsc{\LARGE University of Antwerp}\\[1.5cm] % Name of your university/college
\textsc{\Large }\\[4cm] % Major heading such as course name
\textsc{\Large Scientific Programming}\\[0.5cm] % Minor heading such as course title

%----------------------------------------------------------------------------------------
%	TITLE SECTION
%----------------------------------------------------------------------------------------

\HRule
{ \huge \bfseries Second Session \\ \Large{Exercise 2}}\\ % Title of your document
\HRule \\[1.5cm]

%----------------------------------------------------------------------------------------
%	AUTHOR SECTION
%----------------------------------------------------------------------------------------

\begin{minipage}{0.4\textwidth}
\begin{flushleft} \large
Armin Halilovic - s0122210 % Your name
\end{flushleft}
\end{minipage}
~
\begin{minipage}{0.4\textwidth}
\begin{flushright} \large
\end{flushright}
\end{minipage}\\[4cm]

% If you don't want a supervisor, uncomment the two lines below and remove the section above
%\Large \emph{Author:}\\
%John \textsc{Smith}\\[3cm] % Your name

%----------------------------------------------------------------------------------------
%	DATE SECTION
%----------------------------------------------------------------------------------------

\vfill % Fill the rest of the page with whitespace
{\large August 22, 2016}\\[3cm] % Date, change the \today to a set date if you want to be precise

%----------------------------------------------------------------------------------------
%	LOGO SECTION
%----------------------------------------------------------------------------------------

%\includegraphics{Logo}\\[1cm] % Include a department/university logo - this will require the graphicx package

%----------------------------------------------------------------------------------------


\end{titlepage}
\onecolumn
\tableofcontents
\newpage
%\twocolumn



\section{Problem}
    We are given data points \( (x_i, f_i), i = 0, ..., 20 \) with \( x_i = i - 10 \) and \( f_i = (-1)^i \) and are tasked do the following:
    \begin{enumerate}
        \item Calculate a polynomial interpolant through the data
        \item Calculate the natural cubic spline for the data
        \item Calculate a least squares approximant
        \item Calculate the trigonometric polynomial interpolant
        \item Calculate a trigonometric least squares approximant
    \end{enumerate}

    \begin{figure}[H]
        \centering
        \includegraphics[width=10cm, trim={2cm, 4cm, 2cm, 3cm}, clip]{../build/images/dataPoints.png}
        \caption{The given data points}
        \label{fig:datapoints}
    \end{figure}

    All of this will be done using C++ and the \href{http://www.gnu.org/software/gsl/}{GNU Scientific Library}.
    In the solutions, we briefly describe what we did with excerpts from the code, generate graphs using the interpolation and approximation methods and discuss the results.

\section{Using the code}
    All of the C++ code can be found in the .cpp files and in appendix A of this document.
    main.cpp comes accompanied by createImages.sh, which contains all of the necessary UNIX commands to generate the graph images.
    This file relies on the \href{https://www.gnu.org/software/plotutils/manual/en/html_node/graph.html}{graph} program
    in the GNU plotutils package to plot graphs, so make sure that it is installed.

    To compile and run the program, execute the following commands in the build/ directory:
    \begin{lstlisting}
    cmake ..
    make
    chmod +x ./createImages.sh
    ./data_fitting_smoothing_and_fn_approx.bin
    \end{lstlisting}
    All of the graphs should be present in the build/images/ directory. If they are not there, make sure createImages.sh is
    executable with "chmod +x" and run it to create them.

\section{Solutions}

\subsection{Polynomial interpolant}
\label{sec:firstpoly}
    We use Newton's divided differences interpolation to do the polynomial interpolation through the data. \\

    To execute the interpolation, we need to initialize a couple of GSL structs.
    \begin{lstlisting}
    gsl_interp_accel *acc = gsl_interp_accel_alloc();
    gsl_spline *interp_poly = gsl_spline_alloc(gsl_interp_polynomial, m);
    gsl_spline_init(interp_poly, x_i, y_i, m);
    \end{lstlisting}
    The \href{https://www.gnu.org/software/gsl/manual/html_node/1D-Higher_002dlevel-Interface.html#g_t1D-Higher_002dlevel-Interface}
    {gsl\_spline} functions do not mean we are working with splines, despite what their name may suggest.
    They simply provide a higher level interface so that we can write less code later on. \\ The type of the interpolation
    we will work with is passed into the gsl\_spline\_alloc function. In this case, gsl\_spline\_init  will
    use gsl\_poly\_dd\_init to initialize the polynomial we will use. gsl\_poly\_dd\_init uses Newton's divided differences to build the polynomial,
    which is what we need to calculate new interpolation points. \\

    Now, we can plot a graph.
    \begin{lstlisting}
    double interpValue;
        gsl_interp_accel *acc = gsl_interp_accel_alloc();
        for (double x = minArray(x_i, m); x <= maxArray(x_i, m); x += 0.001) {
            interpValue = gsl_spline_eval(interp_poly, x, acc);
            polynomialInterp << x << " " << interpValue << "\n";
        }
    \end{lstlisting}
    The result of the interpolation can be seen in figure~\ref{fig:poly1}.
    %Graphs showing the relative differences can be found in the .zip file this report came in, they are omitted here as they do not bring


    \begin{figure}[H]
        \begin{minipage}[b]{0.49\textwidth}
            \includegraphics[width=9.25cm, trim={1cm, 4cm, 2cm, 3cm}, clip]{../build/images/polynomialInterp.png}
        \end{minipage}
        \hfill
        \begin{minipage}[b]{0.49\textwidth}
            \includegraphics[width=9.25cm, trim={2cm, 4cm, 2cm, 3cm}, clip]{../build/images/polynomialInterp_zoomed_2.png}
        \end{minipage}
        \caption{Polynomial interpolation through the given data points. Zoomed in image on the right.}
        \label{fig:poly1}
    \end{figure}

    The polynomial interpolant is of degree 20. We see that near the endpoints of the interpolation, the approximations
    become very large values. This happens because of high degree of the polynomial together with the fact that
    we are using equidistant points.
    A way to improve this interpolation would be using Chebyshev nodes as the \( x_i \) points.
    These points are spread out more towards the endpoints of the interval, which would reduce this extreme behaviour.

\subsection{Natural cubic spline}
    For natural cubic spline interpolation, the only difference in the code lies in the initialization of the
    interpolation struct. We simply pass gsl\_interp\_cspline into gsl\_spline\_alloc instead of gsl\_interp\_polynomial
    and plot the grap. The result is visible in figure~\ref{fig:spline1}.
    \begin{figure}[H]
        \centering
        \includegraphics[width=10cm, trim={2cm, 4cm, 2cm, 3cm}, clip]{../build/images/cubicSplineNatural.png}
        \caption{Natural cubic spline interpolation through the given data points}
        \label{fig:spline1}
    \end{figure}
    We get a much nicer result than the polynomial interpolation.
    However, we see that near the end points the interpolation goes beyond -1.
    This leads us to believe that for the given data, the end conditions of the cubic spline are not satisfied. \\

    We achieve more favorable results by using a cubic spline with periodic boundary conditions. This is done by
    passing gsl\_interp\_cspline\_periodic into the gsl spline allocator.
    \begin{figure}[H]
        \centering
        \includegraphics[width=10cm, trim={2cm, 4cm, 2cm, 3cm}, clip]{../build/images/cubicSplinePeriodic.png}
        \caption{Periodic cubic spline interpolation through the given data points}
        \label{fig:spline1}
    \end{figure}

\subsection{Least squares approximation}
To get a least squares approximation, we find an optimal solution for the equation
    \[ \lambda_1 f_1(x_i) + ... +  \lambda_n f_n(x_i) = y_i \ \ where \ \ i = 1,...,m \gg n \]
where m stands for the amount of given data points and n for the amount of unknowns we want to calculate.
This equation can be written as an overdetermined system of equations of the form \(A\lambda = y\).
The system is constructed in the code as follows:
\begin{lstlisting}
gsl_matrix *A = gsl_matrix_alloc(size, n);
gsl_vector *Y = gsl_vector_alloc(size);

for (int i = 0; i < m; i++) {
    for (int j = 0; j < n; j++) {
        gsl_matrix_set(A, i, j, gsl_pow_int(x_rescaled[i], j));
    }
    gsl_vector_set(Y, i, y_i[i]);
}
\end{lstlisting}
We note here that as basis function we used \( f_i(x) = x^i  \).

\subsubsection{QR factorization}
    Since the system \(A \lambda = y \) is overdetermined, we cannot solve it exactly, so we find the least squares solution for it.
    The least squares solution will minimize the Euclidean norm of the residual, \(||A\lambda - y||\). \\
    Before we do that, QR factorization is used to decompose the matrix \( A \) into a product \( A = QR \)
    of an orthogonal matrix \(Q\) and an upper triangular matrix \(R\):
\begin{lstlisting}
gsl_linalg_QR_decomp(QR, tau);
gsl_linalg_QR_lssolve(QR, tau, Y, X, R);
\end{lstlisting}

\subsubsection{Results}
    We calculated the least squares solutions for the system for n = 4, 7, 10, 15, 20, and 21. Detailed output of the solutions for the systems can be found in the output/ directory.
    We show the graphs for these solutions here:
    \begin{figure}[H]
        \begin{minipage}[b]{0.49\textwidth}
            \includegraphics[width=6.5cm, trim={2cm, 4cm, 2cm, 3cm}, clip]{../build/images/leastSquares_degree_4.png}
        \end{minipage}
        \hfill
        \begin{minipage}[b]{0.49\textwidth}
            \includegraphics[width=6.5cm, trim={2cm, 4cm, 2cm, 3cm}, clip]{../build/images/leastSquares_degree_7.png}
        \end{minipage}
        \caption{Approximations for n = 4 and n = 7}
        \label{fig:results1}
    \end{figure}
    \begin{figure}[H]
        \begin{minipage}[b]{0.49\textwidth}
            \includegraphics[width=6.5cm, trim={2cm, 4cm, 2cm, 3cm}, clip]{../build/images/leastSquares_degree_10.png}
        \end{minipage}
        \hfill
        \begin{minipage}[b]{0.49\textwidth}
            \includegraphics[width=6.5cm, trim={2cm, 4cm, 2cm, 3cm}, clip]{../build/images/leastSquares_degree_15.png}
        \end{minipage}
       \caption{Approximations for n = 10 and n = 15}
        \label{fig:results2}
    \end{figure}
    \begin{figure}[H]
        \begin{minipage}[b]{0.49\textwidth}
            \includegraphics[width=6.5cm, trim={2cm, 4cm, 2cm, 3cm}, clip]{../build/images/leastSquares_degree_20_zoomed_out.png}
        \end{minipage}
        \hfill
        \begin{minipage}[b]{0.49\textwidth}
            \includegraphics[width=6.5cm, trim={2cm, 4cm, 2cm, 3cm}, clip]{../build/images/leastSquares_degree_20.png}
        \end{minipage}
        \caption{Approximation for n = 20. Zoomed in image on the right}
        \label{fig:results3}
    \end{figure}
    \begin{figure}[H]
        \begin{minipage}[b]{0.49\textwidth}
            \includegraphics[width=6.5cm, trim={2cm, 4cm, 2cm, 3cm}, clip]{../build/images/leastSquares_degree_21_zoomed_out.png}
        \end{minipage}
        \hfill
        \begin{minipage}[b]{0.49\textwidth}
            \includegraphics[width=6.5cm, trim={2cm, 4cm, 2cm, 3cm}, clip]{../build/images/leastSquares_degree_21.png}
        \end{minipage}
        \caption{Approximation for n = 21. Zoomed in image on the right}
        \label{fig:results4}
    \end{figure}
    Note that when n = 21, we get the same result as with the polynomial interpolation.
    This happens because at n = 21 we are solving the same 21x21 system that the polynomial interpolation solves.
    Thus, at n = 21 we are doing interpolation.

\subsubsection{Condition number of \(A\)}
    We calculate the condition number of \(A\) at each approximation to have some kind of measure for the accuracy of our solutions.
    The condition number is not a direct representation of the accuracy of the solutions, but it plays a big role in it.
    A condition number with a high order of magnitude means that a small change in the input matrix \(A\) \emph{could}
    cause a significant change in the solutions, whereas the opposite counts for condition numbers with smaller orders of magnitude.\\
    To calculate the condition number of A, we use the following definition: \[\kappa(A)= \frac{|max(S)|}{|min(S)|} \]
    where S is the vector of singular values of A. In GSL, we can find S by using gsl\_linalg\_SV\_decomp.
\begin{lstlisting}
gsl_linalg_SV_decomp(U, V, S, work);
double condNumber, minS, maxS;
minS = gsl_vector_get(S, 0);
maxS = gsl_vector_get(S, 0);
for (int j = 0; j < n; j++) {
    if (gsl_vector_get(S, j) < minS) minS = gsl_vector_get(S, j);
    if (gsl_vector_get(S, j) > maxS) maxS = gsl_vector_get(S, j);
}
condNumber = fabs(maxS) / fabs(minS);
\end{lstlisting}
    We find the following condition numbers
\begin{lstlisting}
	n = 4:   Condition number: 1.6514
	n = 7:   Condition number: 91.935
	n = 10: Condition number: 1294.9
	n = 15: Condition number: 1.8592e+05
	n = 20: Condition number: 1.2585e+08
	n = 21: Condition number: 8.3138e+08
\end{lstlisting}
More details for each system can be found in the output/leastSquares\_degree\_*.txt files

A way to improve our results and lower the condition numbers would be to use the Legendre polynomials as basis functions instead
of \( f_i(x) = x^i \), but that was not done in these solutions.


\subsection{Trigonometric polynomial interpolant}
We find a trigonometric polynomial interpolant by using a function of the form:
    \[ f(t) = \frac{a_0}{2} + a_1 \cos (t) + a_2 \cos(2t) + ... + a_m \cos(mt) + b_1 \sin (t) + b_2 \sin (2t) + ... + b_m \sin (mt) \]
Because the given data points are evenly spaced, we can use a simple function to calculate the unknown \( a_i \)'s and \(b_i \)'s:
\begin{lstlisting}
double fourierCoefficient(int j, int n, double *t_i, double *x_i, bool a) {
    double result = 0;

    for (int k = 0; k < n; k++) {
        result += x_i[k] * ((a) ? gsl_sf_cos(j * t_i[k]) : gsl_sf_sin(j * t_i[k]));
    }

    result *= 2.0/(double) n;
    return result;
}
\end{lstlisting}
We fill in the function with \(n = amount \ of \ given\  data \ points = 21\) and \(m = 10\). With \(m = 10\), there are \(2m+1 = 21\)
coefficients, so we can do interpolation.
Using the function, we get the following graph:
    \begin{figure}[H]
        \centering
        \includegraphics[width=10cm, trim={1cm, 3cm, 1cm, 2cm}, clip]{../build/images/trigApprox_n_21_m_10.png}
        \caption{Trigonometric interpolation through the given data points}
        \label{fig:spline1}
    \end{figure}


\subsection{Trigonometric least squares approximant}
We use the same function
    \[ f(t) = \frac{a_0}{2} + a_1 \cos (t) + a_2 \cos(2t) + ... + a_m \cos(mt) + b_1 \sin (t) + b_2 \sin (2t) + ... + b_m \sin (mt) \]
with the same n = 21. This time we choose m such that \( n > 2m+1 \) holds true. The choice of an even vs. an uneven amount
of terms does not matter here. The only time it matters is when \( n = 2m \). In that case, the function \( f(t) \) changes slightly.
Using the function, we get the following graphs for m = 2, 4, and 7:
    \begin{figure}[H]
        \begin{minipage}[b]{0.49\textwidth}
            \includegraphics[width=6.5cm, trim={2cm, 4cm, 2cm, 3cm}, clip]{../build/images/trigApprox_n_21_m_2.png}
        \end{minipage}
        \hfill
        \begin{minipage}[b]{0.49\textwidth}
            \includegraphics[width=6.5cm, trim={2cm, 4cm, 2cm, 3cm}, clip]{../build/images/trigApprox_n_21_m_4.png}
        \end{minipage}
        \caption{Trigonometric least squares approximation for the given data points with m = 2 and 4}
        \label{fig:results4}
    \end{figure}
    \begin{figure}[H]
        \begin{minipage}[b]{0.49\textwidth}
            \includegraphics[width=6.5cm, trim={2cm, 4cm, 2cm, 3cm}, clip]{../build/images/trigApprox_n_21_m_7.png}
        \end{minipage}
        \hfill
        \begin{minipage}[b]{0.49\textwidth}
            \includegraphics[width=6.5cm, trim={2cm, 4cm, 2cm, 3cm}, clip]{../build/images/trigApprox_n_21_m_9.png}
        \end{minipage}
        \caption{Trigonometric least squares approximation for the given data points with m = 7 and 9}
        \label{fig:results4}
    \end{figure}


\onecolumn
\appendix
\appendixpage
\addappheadtotoc

\section{main.cpp}
    \lstinputlisting[basicstyle=\scriptsize]{../main.cpp}
    \newpage

\section{functions.h}
    \lstinputlisting[basicstyle=\scriptsize]{../functions.h}

\bigskip
\section{functions.cpp}
    \lstinputlisting[basicstyle=\scriptsize]{../functions.cpp}

\section{createImages.sh}
    \lstinputlisting[language=bash, basicstyle=\scriptsize]{../createImages.sh}

\end{document}