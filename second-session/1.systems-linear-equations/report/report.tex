documentclass[11pt, a4paper, titlepage, openright]{article}

\usepackage{amsmath}
\usepackage[font=small,labelfont=bf]{caption}
\usepackage{float}
\restylefloat{figure}
\usepackage{graphicx}
\usepackage{hyperref}
\usepackage{mathtools}
\usepackage[titletoc, title]{appendix}
\usepackage{listings}
\usepackage{color}
\usepackage{fixltx2e}
\usepackage[bottom]{footmisc}

\usepackage[a4paper, total={6.5in, 9.5in}]{geometry}

\definecolor{dkgreen}{rgb}{0,0.6,0}
\definecolor{gray}{rgb}{0.5,0.5,0.5}
\definecolor{mauve}{rgb}{0.58,0,0.82}

\lstset{frame=tb,
  language=C++,
  aboveskip=3mm,
  belowskip=3mm,
  showstringspaces=false,
  columns=flexible,
  basicstyle={\footnotesize\ttfamily},
  numbers=none,
  numberstyle=\tiny\color{gray},
  keywordstyle=\color{blue},
  commentstyle=\color{dkgreen},
  stringstyle=\color{mauve},
  breaklines=true,
  breakatwhitespace=true,
  tabsize=3,
  showstringspaces=false,
  keepspaces=true, 
  columns=flexible
  }

\begin{document}
\input{title_page.tex}
\tableofcontents
\newpage

\section{Problem}
    We have to solve the system of linear equations
        \[
            \begin{aligned}
                \sum_{j=1}^{n} (1+i)^{j-1} x_j = \frac{(1+i)^{n} - 1}{i}     && 1 \le i \le n
            \end{aligned}
        \]
    for \( n = 10, ..., 15 \) using Gauss-elimination with partial pivoting.
    Then, we calculate and discuss the results and the condition numbers of the coeffient matrices of the linear systems. \\
    This is done using C++ and the \href{http://www.gnu.org/software/gsl/}{GNU Scientific Library}.
    In section~\ref{sec:solutions}, we explain how the solutions are reached, using \emph{only} the
    most important parts from our code. Some basic knowledge of GSL is assumed.
\bigskip
\bigskip

\section{Using the program}
    All of the C++ code for the program can be found in the main.cpp, functions.cpp and functions.h files, 
    and in appendix A of this document. The output of the program is stored in output/output\_n\_*.txt. The output file for \(n=10\) can be found in appendix B.

    To compile and run the program, execute the following commands in the build/ directory:
\begin{lstlisting}
cmake ..
make
./sys_lineq.bin
\end{lstlisting}


\newpage
\section{Course of action}
    The systems of linear equations have the form \(Ax = y\).
    For example, if n = 10, we get:
    \[
        A =
        \begin{bmatrix}
            %(1+i)^{0} (1+i)^{1} (1+i)^{2} (1+i)^{3} (1+i)^{4} (1+i)^{5} (1+i)^{6} (1+i)^{7} (1+i)^{8} (1+i)^{9} (1+i)^{10} (1+i)^{11} (1+i)^{12} (1+i)^{13} (1+i)^{14}
            (1+i)^{0} & (1+i)^{1} & ... & (1+i)^{9} \\
            (1+i)^{0} & (1+i)^{1} & ... & (1+i)^{9} \\
            (1+i)^{0} & (1+i)^{1} & ... & (1+i)^{9} \\
            (1+i)^{0} & (1+i)^{1} & ... & (1+i)^{9} \\
            (1+i)^{0} & (1+i)^{1} & ... & (1+i)^{9} \\
            (1+i)^{0} & (1+i)^{1} & ... & (1+i)^{9} \\
            (1+i)^{0} & (1+i)^{1} & ... & (1+i)^{9} \\
            (1+i)^{0} & (1+i)^{1} & ... & (1+i)^{9} \\
            (1+i)^{0} & (1+i)^{1} & ... & (1+i)^{9} \\
            (1+i)^{0} & (1+i)^{1} & ... & (1+i)^{9} \\
        \end{bmatrix}
        x =
        \begin{bmatrix}
            \\
            x_{1} \\\\
            x_{2} \\\\
            x_{3} \\\\
            . \\\\
            . \\\\
            . \\\\
            x_{10} \\\\
        \end{bmatrix}
        \\
        y =
        \begin{bmatrix}
            \\
            \frac{(1+i)^{10} - 1}{i} \\\\
            \frac{(1+i)^{10} - 1}{i} \\\\
            \frac{(1+i)^{10} - 1}{i} \\\\
            . \\\\
            . \\\\
            . \\\\
            \frac{(1+i)^{10} - 1}{i} \\\\
        \end{bmatrix}
    \]
    The i's in the systems can be chosen freely, as long as the constraint \( 1 \le i \le n \) holds true.
    We choose the i's to be the row indices of the systems. \\
    What follows is a brief set of steps we use to solve the matrices.
    \begin{enumerate}
        \item Load the input matrix A and vector Y.
        \item Find the LU decomposition of A.
                    \subitem Gauss-elimination with partial pivoting happens here.
        \item Solve the system using the LU decomposition.
            \subitem Also, find the residual error vector.\footnote{We cannot find the error vector, since we do not have the exact solution for \(x\) in \(Ax = y\)}
        \item Calculate the condition number of the matrix.
    \end{enumerate}
    
\newpage
\section{Solutions}
\label{sec:solutions}
    \subsection{Solution for n=10}
        Initializing all of the necessary structs and loading the input matrix and vector
        is a trivial task, so we start at the LU decomposition step.
    \subsubsection{LU decomposition}
    To find LU decompositions, we use gsl\_linalg\_LU\_decomp. This function makes use of Gaussian Elimination with partial pivoting to find the LU decompositions.
\begin{lstlisting}
gsl_matrix_memcpy(LU, A);
gsl_linalg_LU_decomp(LU, P, &sigNum);
\end{lstlisting}
    gsl\_linalg\_LU\_decomp writes the result to its first argument, so we copy A to a new matrix
    LU first, as we need A later on. P contains the permutation matrix, which is not relevant for
    this exercise. \\ The resulting matrix LU can be found in the output file for n = 10.

    \subsubsection{Solving the system}
    The code for this part is very straightforward:
\begin{lstlisting}
gsl_linalg_LU_solve(LU, P, Y, X);
gsl_linalg_LU_refine(A, LU, P, Y, X, r);
\end{lstlisting}
    gsl\_linalg\_LU\_solve uses the LU decomposition and permutation matrix to solve the system of equations.
    The solution for the system is stored in X. gsl\_linalg\_LU\_refine is then used to find the residual vector r.\\
    The residual vector is the vector r so that \[ r = y - Ax \] where \(y\) is known exactly and \(Ax\) is calculated. 
    
    We get the output:
\begin{lstlisting}
Vector X, result of solving with LU decomposition:
            1
            1
            1
            1
            1
            1
            1
            1
            1
            1

Vector r, residual of solving by LU decomposition:
   4.3321e-05
  -8.1093e-05
   6.4058e-05
  -2.8163e-05
   7.6338e-06
  -1.3279e-06
   1.4868e-07
   -1.036e-08
   4.0865e-10
   -6.966e-12
\end{lstlisting}
    We notice that the values for the residual vector are very small, this indicates that the resulting
    vector X is very precise. This does not necessarily mean that the solution to the system is exactly this vector.
    However, if we check, we see that the formula
    \[
        \begin{aligned}
            \sum_{j=1}^{n} (1+i)^{j-1} x_j = \frac{(1+i)^{n} - 1}{i}     && 1 \le i \le n
        \end{aligned}
    \]
    indeed is true when all x\_j are 1.

    \subsubsection{Condition number of A}
    To calculate the condition number of A, we use the following definition: \[\kappa(A)= \frac{|max(S)|}{|min(S)|} \]
    where S is the vector of singular values of A. In GSL, we can find S by using gsl\_linalg\_SV\_decomp.
\begin{lstlisting}
gsl_linalg_SV_decomp(U, V, S, work);

double condNumber, minS, maxS;
minS = gsl_vector_get(S, 0); maxS = gsl_vector_get(S, 0);
for (int j = 0; j < size_j; j++) {
    if (gsl_vector_get(S, j) < minS) minS = gsl_vector_get(S, j);
    if (gsl_vector_get(S, j) > maxS) maxS = gsl_vector_get(S, j);
}
condNumber = fabs(maxS) / fabs(minS);
\end{lstlisting}
    We get the following output:
\begin{lstlisting}
Vector S, singular values of A, result of doing SV decomposition:
   2.6054e+09
    1.225e+07
   1.4309e+05
       3220.7
       130.72
       9.8208
       1.2767
      0.13121
    0.0059818
   9.6998e-05

Condition number = 2.6054e+09 / 9.6998e-05 = 2.6861e+13
\end{lstlisting}
    We get a condition number of about 2.7e+13. \\
    Informally, we know that the larger the order of the condition number is, the larger the possible error (in the X vector) can be.
    For very large condition numbers, a small pertubation in one of the matrix' cells could cause a large change in the results.
    We test and compare the effect of small perturbations later on.

%    Next, we use the condition number and more formally describe the maximum inaccuracy which \emph{can} occur after all of the calculations.
%
%    \subsubsection{Maximum possible error}
%    We know the following inequalitíes which express that the maximum possible relative errors are bounded:
%    \[\frac{\mid\mid y - Ax\textsubscript{*} \mid\mid}{\mid\mid A \mid\mid \ \mid\mid x\textsubscript{*}\mid\mid} \leq \rho \epsilon,\ \ \ \ \
%    \frac{\mid\mid x - x\textsubscript{*} \mid\mid}{\mid\mid x\textsubscript{*} \mid\mid} \leq \rho \kappa(A) \epsilon \]
%    Here, \(\epsilon\) is the relative machine precision eps. For a double,
%    this is \(2.22\mbox{\sc{e}-}16 \).\footnote{https://en.wikipedia.org/wiki/Machine\_epsilon\#Values\_for\_standard\_hardware\_floating\_point\_arithmetics}
%    If we take the approximately worst case for p, which is around 10, then we get that \(\rho \epsilon = 2.22\mbox{\sc{e}-}15 \)
%
%    In the second formula, we see a better way to describe the impact that the condition number has on the
%    accuracy of the calculations for \(x\). For \(\rho \kappa(A) \epsilon\), we find:
%\begin{align*}
%    \rho \kappa(A) \epsilon \approx 10 * 2.69\mbox{\sc{e}+}13 * 2.22\mbox{\sc{e}-}16\\
%    \rho \kappa(A) \epsilon \approx 2.69 * 2.22\mbox{\sc{e}-}2\\
%    \rho \kappa(A) \epsilon \approx 5.97\mbox{\sc{e}-}2
%\end{align*}
%     We can now say that our result for \(x\) has a maximal error of \(5.97\mbox{\sc{e}-}2\),
%     with respect to the exact solutions for \(x\).



\subsection{Solution for n=11}
    We get the following results
\begin{lstlisting}
Vector X, result of solving with LU decomposition:
        1.001
      0.99808
       1.0016
       0.9992
       1.0002
      0.99995
            1
            1
            1
            1
            1

Vector r, residual of solving by LU decomposition:
   -0.0013454
    0.0027319
   -0.0023955
    0.0011972
  -0.00037823
   7.9061e-05
  -1.1093e-05
   1.0335e-06
  -6.1298e-08
   2.0934e-09
   -3.131e-11

Condition number = 6.8216e+10 / 4.1701e-05 = 1.6358e+15
\end{lstlisting}
    We notice that the condition number has increased, which means the possible error that \emph{may} occur also increased.
    %We see this when calculating \(\rho \kappa(A) \epsilon\):
    %\begin{align*}
    %    \rho \kappa(A) \epsilon \approx 10 * 1.64\mbox{\sc{e}+}15 * 2.22\mbox{\sc{e}-}16\\
    %    \rho \kappa(A) \epsilon \approx 1.64 * 2.22\\
    %    \rho \kappa(A) \epsilon \approx 3.64
    %\end{align*}


\subsection{Solution for n=12}
    We get the following results
\begin{lstlisting}
Vector X, result of solving with LU decomposition:
      0.97184
       1.0587
      0.94665
        1.028
      0.99058
       1.0021
      0.99966
            1
            1
            1
            1
            1

Vector r, residual of solving by LU decomposition:
     0.065446
     -0.13345
      0.11782
    -0.059773
     0.019457
   -0.0042842
   0.00065322
  -6.9152e-05
   4.9923e-06
  -2.3452e-07
    6.462e-09
  -7.9229e-11

Condition number = 1.9698e+12 / 1.1463e-05 = 1.7184e+17
\end{lstlisting}
    The condition number has increased again.




\subsection{Solution for n=13}
    We get the following results
\begin{lstlisting}
Vector X, result of solving with LU decomposition:
       2.6882
        -2.65
       4.4667
     -0.91978
       1.6926
      0.82808
       1.0302
      0.99622
       1.0003
      0.99998
            1
            1
            1

Vector r, residual of solving by LU decomposition:
      -66.275
       142.06
      -133.52
       73.034
      -25.974
       6.3446
      -1.0941
      0.13448
    -0.011716
   0.00070686
  -2.8076e-05
   6.6017e-07
  -6.9591e-09

Condition number = 6.2191e+13 / 6.1062e-05 = 1.0185e+18
\end{lstlisting}
    The condition number has increased again.




\subsection{Solution for n=14}
    We get the following results
\begin{lstlisting}
Vector X, result of solving with LU decomposition:
      0.17084
       2.3299
      0.30066
      0.98717
       1.1907
      0.89236
        1.033
      0.99344
       1.0009
      0.99992
            1
            1
            1
            1

Vector r, residual of solving by LU decomposition:
      -941.62
         2021
      -1911.7
       1060.2
      -386.08
       97.825
      -17.794
       2.3584
     -0.22827
     0.015965
  -0.00078561
   2.5795e-05
  -5.0724e-07
   4.5171e-09

Condition number = 2.1314e+15 / 0.00017035 = 1.2512e+19
\end{lstlisting}
    The condition number has increased again.




\subsection{Solution for n=15}
    We get the following results
\begin{lstlisting}
Vector X, result of solving with LU decomposition:
      -6721.2
        15434
       -15775
       9552.4
      -3837.3
       1086.8
      -222.48
       35.065
      -2.8712
       1.3269
      0.97978
       1.0009
      0.99997
            1
            1

Vector r, residual of solving by LU decomposition:
       -12716
        28066
       -27411
        15768
      -5991.5
       1596.1
      -308.34
        44.01
      -4.6765
      0.36907
    -0.021347
    0.0008794
   -2.444e-05
   4.1093e-07
  -3.1584e-09

Condition number = 7.8805e+16 / 0.00074503 = 1.0577e+20
\end{lstlisting}
    The condition number has increased again.





\newpage
\section{Discussion}
We notice that when n increases, so does the condition number of each system.
This makes sense, as larger and larger numbers are added to the system while the rest of the system stays the same.
The scale of the system becomes larger each time.\\
As a result of this, at each increment of n, X seems to move farther away from the vector of all ones. The residual vector becomes worse each time as well.\\

For each n, we looked at the result of changing the cell (10, 2) of the input matrix to 1. \\
For n = 10, we found:
\begin{lstlisting}
Vector X, result of solving with LU decomposition:
       1.5211
   -0.0052109
       1.8263
      0.61862
       1.1093
      0.97977
       1.0024
      0.99982
            1
            1
\end{lstlisting}

For n = 15, we found:
\begin{lstlisting}
Vector X, result of solving with LU decomposition:
       9.2501
   -0.0060889
      -19.703
       28.878
      -17.302
       8.4653
      -1.0681
       1.4059
      0.94242
       1.0059
      0.99956
            1
            1
            1
            1
\end{lstlisting}

We see that the small perturbation cause a relatively small change in the solution when n = 10, where the condition number is relatively small compared to the other ones.
On the other hand, the small perturbation caused a very large change in the solution where n = 15.
If we look at the l\textsubscript{2} norms for example, the norm changed from 3.16 to 3.50 when n = 10, and changed from 25287.12 to 41.10 when n = 15.

\onecolumn
\appendix
\appendixpage
\addappheadtotoc

\section{Code}
	\subsection{main.cpp}
		\lstinputlisting[basicstyle=\scriptsize]{../main.cpp} 
	\bigskip
	\subsection{functions.h}
		\lstinputlisting[basicstyle=\scriptsize]{../functions.h}
	\bigskip
	\subsection{functions.cpp}
		\lstinputlisting[basicstyle=\scriptsize]{../functions.cpp}

\section{Output}
	\subsection{output\_n\_10.txt}
		\lstinputlisting[basicstyle=\scriptsize]{../build/output_n_10.txt}
\newpage
		
\end{document}