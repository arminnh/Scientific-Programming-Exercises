\documentclass[11pt, a4paper, titlepage, openright]{article}

\usepackage{amsmath}
\usepackage[font=small,labelfont=bf]{caption}
\usepackage{float}
\restylefloat{figure}
\usepackage{graphicx}
\usepackage{hyperref}
\usepackage{mathtools}
\usepackage[titletoc, title]{appendix}
\usepackage{listings}
\usepackage{color}
\usepackage{fixltx2e}
\usepackage[bottom]{footmisc}

\definecolor{dkgreen}{rgb}{0,0.6,0}
\definecolor{gray}{rgb}{0.5,0.5,0.5}
\definecolor{mauve}{rgb}{0.58,0,0.82}

\lstset{frame=tb,
  language=C++,
  aboveskip=3mm,
  belowskip=3mm,
  showstringspaces=false,
  columns=flexible,
  basicstyle={\footnotesize\ttfamily},
  numbers=none,
  numberstyle=\tiny\color{gray},
  keywordstyle=\color{blue},
  commentstyle=\color{dkgreen},
  stringstyle=\color{mauve},
  breaklines=true,
  breakatwhitespace=true,
  tabsize=3,
  showstringspaces=false,
  keepspaces=true, 
  columns=flexible
}

\begin{document}
\input{title_page.tex}
\tableofcontents
\newpage

\section{Problem}
    We are given the following set of data function:
    \[ f(x) = \sqrt{1 + \frac{x}{5}},\ where \ x \ \epsilon \ [-1, 1]\]

    We will approximate this function using fourier method:
    \[ g(t) = \frac{a_0}{2} + a_1 \cos (t) + a_2 \cos(2t) + ... + a_n \cos(nt) + b_1 \sin (t) + b_2 \sin (2t) + ... + b_m \sin (nt) \]


    This will be done using C++ and the \href{http://www.gnu.org/software/gsl/}{GNU Scientific Library}.
    In section~\ref{sec:solutions}, we will describe how we reached each solution, using the 
    most important parts from the code.

\bigskip
\bigskip
\section{Using the program}
    All of the C++ code for the program can be found in the main.cpp and in appendix A of this document.
    main.cpp comes accompanied by \mbox{function\_approximation.sh}, which contains all of the necessary UNIX commands to generate the graph images.
    This file relies on the \href{https://www.gnu.org/software/plotutils/manual/en/html_node/graph.html}{graph} program
    in the GNU plotutils package to plot graphs, so make sure that it is installed.

    To compile and run the program, execute the following commands in the build/ directory:
\begin{lstlisting}
cmake ..
make
chmod +x ./function_approximation.sh
./function_approximation.bin
\end{lstlisting}
    Do not forget the .bin extension. All of the graphs will be present in the build/images/ directory. If no graphs are present, they can be generated
    manually by executing function\_approximation.sh. \\
    Output files which give some extra information about the solutions can be found in console output  and appendix B.1.
    
\newpage
\section{Solutions}
\label{sec:solutions}
    In these solutions, m denotes the amount of data points we will use to approximate \( f(x) \) with.
    \subsection{Calculating the data points}
    We will generate m equidistant data points in the interval \([-1, 1]\), using the formula \( -1 + \frac{2*(i-1)}{m} \).  \\
    The array \(t\) will hold these points, array \(x\) will hold \( f(t) \). The points \(t_i\) wil then be rescaled to lie in \( [-\pi, \pi] \):
\begin{lstlisting}
double t[m], x[m];

for (int i = 1; i <= m; i++) {
    t[i-1] = -1 + (2*(i-1) / (double) (m-1));
    x[i-1] = fx(t[i-1]);
    t[i-1] = M_PI * t[i-1];
}
\end{lstlisting}
    \subsection{Calculating the coefficients}
    To use the approximating formula, we first need to find the coefficients \( a_0, a_1, b_1, a_2, b_2, ... \).
    The formulas for the coefficients are:
    \[ a_j = \frac{2}{m} \sum_{k=0}^{m-1}x_k \cos(j*t_k),\ \  b_j = \frac{2}{m} \sum_{k=0}^{m-1}x_k \sin(j*t_k) \]
    where \( j = 0, ..., n \) and \( x_k \) are the given data points we work with.
    In the code, this is done by the function "calcCoeff":
\begin{lstlisting}
double calcCoeff(double *t, double *x, bool a, int j, int m) {
    double result = 0;

    for (int k = 0; k < m; k++) {
        result += x[k] * ((a) ? gsl_sf_cos(j*t[k]) : gsl_sf_sin(j*t[k]));
    }

    result *= 2.0/(double) m;
    return result;
}
\end{lstlisting}
    \newpage
    \subsection{Approximating the function}
    We can now fill in and use the function
    \[ g(t) = \frac{a_0}{2} + a_1 \cos (t) + a_2 \cos(2t) + ... + a_n \cos(nt) + b_1 \sin (t) + b_2 \sin (2t) + ... + b_m \sin (nt) \]
    to approximate \(f(x)\):
    \\
\begin{lstlisting}
for (double t = -M_PI; t < M_PI; t = t + 0.01) {
    double y = coeff[0]; //y = a_0/2
    for (int i = 0; i < n; i++) {
        int j = 2*i+1;
        y += coeff[j]   * gsl_sf_cos((i+1)*t);
        y += coeff[j+1] * gsl_sf_sin((i+1)*t);
    }
}
\end{lstlisting}
    
    The approximation functions \( f(t) \) can be found in Appendix B.1. 
    The graphs that are generated can be found in Appendix B.2 and B.3. The graphs in Appendix B.3 are approximations for \( f(x) = |x| \).
    
    We notice in both cases that, generally, the approximation becomes more accurate with an increase in m or n. This is to be expected with fourier methods.

\onecolumn
\appendix
\appendixpage
\addappheadtotoc

\section{Code}
	\subsection{main.cpp}
		\lstinputlisting[basicstyle=\scriptsize]{../main.cpp}
	\bigskip
	\subsection{function\_approximation.sh}
		\lstinputlisting[basicstyle=\scriptsize, language=bash]{../build/function_approximation.sh}

\newpage
\section{Output}
\label{sec:output}
	\subsection{Console output}
	\label{sec:consoleOut}
		\lstinputlisting[basicstyle=\scriptsize]{../consoleOutput.txt}

	\subsection{\( f(x) = \sqrt{1 + \frac{x}{5}} \)}
	\label{sec:sqrtImages}
    \begin{figure}[H]
        \begin{minipage}[b]{0.49\textwidth}
            \includegraphics[width=6.5cm, trim={2cm, 4cm, 2cm, 3cm}, clip]{../build/images/fx}
        \end{minipage}
        \hfill
        \begin{minipage}[b]{0.49\textwidth}
            \includegraphics[width=6.5cm, trim={2cm, 4cm, 2cm, 3cm}, clip]{../build/images/m3n1}
        \end{minipage}
        \caption{left: f(x), right: approximation where m=3 and n=1}
        \label{fig:sqrt1}
    \end{figure}
    \begin{figure}[H]
        \begin{minipage}[b]{0.49\textwidth}
            \includegraphics[width=6.5cm, trim={2cm, 4cm, 2cm, 3cm}, clip]{../build/images/m5n1}
        \end{minipage}
        \hfill
        \begin{minipage}[b]{0.49\textwidth}
            \includegraphics[width=6.5cm, trim={2cm, 4cm, 2cm, 3cm}, clip]{../build/images/m10n1}
        \end{minipage}
        \caption{approximations where m=5, n=1, and m=10, n=1}
        \label{fig:sqrt1}
    \end{figure}
    \begin{figure}[H]
        \begin{minipage}[b]{0.49\textwidth}
            \includegraphics[width=6.5cm, trim={2cm, 4cm, 2cm, 3cm}, clip]{../build/images/m25n1}
        \end{minipage}
        \hfill
        \begin{minipage}[b]{0.49\textwidth}
            \includegraphics[width=6.5cm, trim={2cm, 4cm, 2cm, 3cm}, clip]{../build/images/m3n3}
        \end{minipage}
        \caption{approximations where m=25, n=1, and m=3, n=3}
        \label{fig:sqrt1}
    \end{figure}
    \begin{figure}[H]
        \begin{minipage}[b]{0.49\textwidth}
            \includegraphics[width=6.5cm, trim={2cm, 4cm, 2cm, 3cm}, clip]{../build/images/m5n3}
        \end{minipage}
        \hfill
        \begin{minipage}[b]{0.49\textwidth}
            \includegraphics[width=6.5cm, trim={2cm, 4cm, 2cm, 3cm}, clip]{../build/images/m10n3}
        \end{minipage}
        \caption{approximations where m=5, n=3, and m=10, n=3}
        \label{fig:sqrt1}
    \end{figure}
    \begin{figure}[H]
        \begin{minipage}[b]{0.49\textwidth}
            \includegraphics[width=6.5cm, trim={2cm, 4cm, 2cm, 3cm}, clip]{../build/images/m25n3}
        \end{minipage}
        \hfill
        \caption{approximation where m=25, n=3}
        \label{fig:sqrt1}
    \end{figure}
    
    \bigskip
    \bigskip
    \bigskip
    \bigskip
	\subsection{Extra: \( f(x) = |x| \)}
	\label{sec:absImages}
    \begin{figure}[H]
        \begin{minipage}[b]{0.49\textwidth}
            \includegraphics[width=6.5cm, trim={2cm, 4cm, 2cm, 3cm}, clip]{../absImages/fx}
        \end{minipage}
        \hfill
        \begin{minipage}[b]{0.49\textwidth}
            \includegraphics[width=6.5cm, trim={2cm, 4cm, 2cm, 3cm}, clip]{../absImages/m3n1}
        \end{minipage}
    \end{figure}
    \begin{figure}[H]
        \begin{minipage}[b]{0.49\textwidth}
            \includegraphics[width=6.5cm, trim={2cm, 4cm, 2cm, 3cm}, clip]{../absImages/m5n1}
        \end{minipage}
        \hfill
        \begin{minipage}[b]{0.49\textwidth}
            \includegraphics[width=6.5cm, trim={2cm, 4cm, 2cm, 3cm}, clip]{../absImages/m10n1}
        \end{minipage}
    \end{figure}
    \begin{figure}[H]
        \begin{minipage}[b]{0.49\textwidth}
            \includegraphics[width=6.5cm, trim={2cm, 4cm, 2cm, 3cm}, clip]{../absImages/m25n1}
        \end{minipage}
        \hfill
        \begin{minipage}[b]{0.49\textwidth}
            \includegraphics[width=6.5cm, trim={2cm, 4cm, 2cm, 3cm}, clip]{../absImages/m3n3}
        \end{minipage}
    \end{figure}
    \begin{figure}[H]
        \begin{minipage}[b]{0.49\textwidth}
            \includegraphics[width=6.5cm, trim={2cm, 4cm, 2cm, 3cm}, clip]{../absImages/m5n3}
        \end{minipage}
        \hfill
        \begin{minipage}[b]{0.49\textwidth}
            \includegraphics[width=6.5cm, trim={2cm, 4cm, 2cm, 3cm}, clip]{../absImages/m10n3}
        \end{minipage}
    \end{figure}
    \begin{figure}[H]
        \begin{minipage}[b]{0.49\textwidth}
            \includegraphics[width=6.5cm, trim={2cm, 4cm, 2cm, 3cm}, clip]{../absImages/m25n3}
        \end{minipage}
        \hfill
    \end{figure}

\end{document}