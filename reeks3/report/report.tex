\documentclass[11pt, a4paper, titlepage, openright]{article}

\usepackage{amsmath}
\usepackage[font=small,labelfont=bf]{caption}
\usepackage{float}
\restylefloat{figure}
\usepackage{graphicx}
\usepackage{hyperref}
\usepackage{mathtools}
\usepackage[titletoc, title]{appendix}
\usepackage{listings}
\usepackage{color}
\usepackage{fixltx2e}
\usepackage[bottom]{footmisc}

\definecolor{dkgreen}{rgb}{0,0.6,0}
\definecolor{gray}{rgb}{0.5,0.5,0.5}
\definecolor{mauve}{rgb}{0.58,0,0.82}

\lstset{frame=tb,
  language=C++,
  aboveskip=3mm,
  belowskip=3mm,
  showstringspaces=false,
  columns=flexible,
  basicstyle={\footnotesize\ttfamily},
  numbers=none,
  numberstyle=\tiny\color{gray},
  keywordstyle=\color{blue},
  commentstyle=\color{dkgreen},
  stringstyle=\color{mauve},
  breaklines=true,
  breakatwhitespace=true,
  tabsize=3,
  showstringspaces=false,
  keepspaces=true, 
  columns=flexible
}

\begin{document}
%%%%%%%%%%%%%%%%%%%%%%%%%%%%%%%%%%%%%%%%%
% University Assignment Title Page
% LaTeX Template
% Version 1.0 (27/12/12)
%
% This template has been downloaded from:
% http://www.LaTeXTemplates.com
%
% Original author:
% WikiBooks (http://en.wikibooks.org/wiki/LaTeX/Title_Creation)
%
% License:
% CC BY-NC-SA 3.0 (http://creativecommons.org/licenses/by-nc-sa/3.0/)
%
% Instructions for using this template:
% This title page is capable of being compiled as is. This is not useful for
% including it in another document. To do this, you have two options:
%
% 1) Copy/paste everything between \begin{document} and \end{document}
% starting at \begin{titlepage} and paste this into another LaTeX file where you
% want your title page.
% OR
% 2) Remove everything outside the \begin{titlepage} and \end{titlepage} and
% move this file to the same directory as the LaTeX file you wish to add it to.
% Then add \input{./title_page_1.tex} to your LaTeX file where you want your
% title page.
%
%%%%%%%%%%%%%%%%%%%%%%%%%%%%%%%%%%%%%%%%%

%----------------------------------------------------------------------------------------
%	PACKAGES AND OTHER DOCUMENT CONFIGURATIONS
%----------------------------------------------------------------------------------------
\begin{titlepage}

\newcommand{\HRule}{\rule{\linewidth}{0.5mm}} % Defines a new command for the horizontal lines, change thickness here

\center % Center everything on the page

%----------------------------------------------------------------------------------------
%	HEADING SECTIONS
%----------------------------------------------------------------------------------------

\textsc{\LARGE University of Antwerp}\\[1.5cm] % Name of your university/college
\textsc{\Large }\\[4cm] % Major heading such as course name
\textsc{\Large Scientific Programming}\\[0.5cm] % Minor heading such as course title

%----------------------------------------------------------------------------------------
%	TITLE SECTION
%----------------------------------------------------------------------------------------

\HRule
{ \huge \bfseries Second Session \\ \Large{Exercise 2}}\\ % Title of your document
\HRule \\[1.5cm]

%----------------------------------------------------------------------------------------
%	AUTHOR SECTION
%----------------------------------------------------------------------------------------

\begin{minipage}{0.4\textwidth}
\begin{flushleft} \large
Armin Halilovic - s0122210 % Your name
\end{flushleft}
\end{minipage}
~
\begin{minipage}{0.4\textwidth}
\begin{flushright} \large
\end{flushright}
\end{minipage}\\[4cm]

% If you don't want a supervisor, uncomment the two lines below and remove the section above
%\Large \emph{Author:}\\
%John \textsc{Smith}\\[3cm] % Your name

%----------------------------------------------------------------------------------------
%	DATE SECTION
%----------------------------------------------------------------------------------------

\vfill % Fill the rest of the page with whitespace
{\large August 22, 2016}\\[3cm] % Date, change the \today to a set date if you want to be precise

%----------------------------------------------------------------------------------------
%	LOGO SECTION
%----------------------------------------------------------------------------------------

%\includegraphics{Logo}\\[1cm] % Include a department/university logo - this will require the graphicx package

%----------------------------------------------------------------------------------------


\end{titlepage}
\tableofcontents
\newpage

\section{Problem}
    We are given the following set of data points:
    \[
    \begin{tabular}{l | *{7}{c}}
        x & 0.635 & 1.435 & 2.235 & 3.035 & 3.835 & 4.635 & 5.435 \\
        y & 7.50  & 4.35  & 2.97  & 2.20  & 1.70  & 1.28  & 1.00 \\
    \end{tabular}
    \]

    We will use data smoothing techniques to create an approximating function that attempts to capture important patterns in the data.
    We assume that the following is true:
    \[ y_i = \lambda_1 f_1(x_i) + ... +  \lambda_n f_n(x_i) \ \ where \ \ i = 1,...,m \gg n \]
    From this, we can for construct the \(m * n \) linear system \(A\lambda = y \). \\
    We will use QR factorization to decompose and solve this system.\\

    All of this will be done using C++ and the \href{http://www.gnu.org/software/gsl/}{GNU Scientific Library}.
    In section~\ref{sec:solutions}, we will describe how we reached each solution, using \emph{only} the 
    most important parts from our code. Some basic knowledge of the GSL is assumed.

\bigskip
\bigskip
\section{Using the program}
    All of the C++ code for the program can be found in the main.cpp and in appendix A of this document.
    main.cpp comes accompanied by \mbox{createImages.sh}, which contains all of the necessary UNIX commands to generate the graph images.
    This file relies on the \href{https://www.gnu.org/software/plotutils/manual/en/html_node/graph.html}{graph} program
    in the GNU plotutils package to plot graphs, so make sure that it is installed.

    To compile and run the program, execute the following commands in the build/ directory:
\begin{lstlisting}
cmake ..
make
chmod +x ./createImages.sh
./data_smoothing
\end{lstlisting}
    All of the graphs should be present in the build/images/ directory. If they are not there, make createImages.sh executable
    with "chmod +x" and run it to create them.

\newpage
\section{Course of action}
    What follows is a brief set of steps we will use to solve the exercise:
    \begin{enumerate}
        \item Load the input matrix A and vector Y.
        \item Find the LU decomposition of A.
        \item Solve the system using the LU decomposition.
            \subitem Also, find the residual error vector.\footnote{We cannot find the error vector, since we do not have the exact solution for \(x\) in \(Ax = y\)}
        \item Calculate the condition number of the matrix.
        \item Calculate the maximum possible error of the solution.
    \end{enumerate}
    
\bigskip
\section{Solutions}
\label{sec:solutions}
    \subsection{Datapoints plot}
    \subsection{Plot of \(\log(y_i)\)}
    \subsection{Setting up the matrix}
    \subsection{QR factorization}
    \subsection{Solving the first system of equations}
        Initializing all of the necessary structs and loading the input matrix and vector 
        is a trivial task, so we will start at the LU decomposition step.
    \subsubsection{LU decomposition}
    We know that \(PA = LU\). GSL makes it very easy to find LU decompositions; gsl\_linalg\_LU\_decomp will find the LU 
    decomposition by executing Gaussian Elimination with partial pivoting.
\begin{lstlisting}
gsl_matrix_memcpy(LU, A);
gsl_linalg_LU_decomp(LU, P, &sigNum);
\end{lstlisting}
    gsl\_linalg\_LU\_decomp writes the result to its first argument, so we will copy A to a new matrix 
    LU first, as we will need A later on. P will contain the permutation matrix, which is not relevant for 
    this exercise. \\ This will result in the following output:
    
\begin{lstlisting}
Matrix LU, result of LU decomposition:
                5.084               -5.832                9.155
    0.594217151848938     5.63947442958301     1.47294197482297
     0.20279307631786   -0.547978507128854   0.0715699307609119
     
\end{lstlisting}
    \subsubsection{Solving the system}
    The code for this part is very straightforward:
\begin{lstlisting}
gsl_linalg_LU_solve(LU, P, Y, X);
gsl_linalg_LU_refine(A, LU, P, Y, X, r);
\end{lstlisting}
    gsl\_linalg\_LU\_solve will use the LU decomposition and permutation matrix to solve the system of equations.
    The solution for the system will be stored in X. gsl\_linalg\_LU\_refine is then used to find the residual vector r.\\
    The residual vector is the vector r so that \[ r = y - Ax \] where \(y\) is known exactly and \(Ax\) is calculated. 
    
    We will get the output:
\begin{lstlisting}
Vector X, result of solving with LU decomposition:
    -7.57416762700087
   0.0158780881981559
     5.13453514211295

Vector r, residual of solving by LU decomposition:
  4.0146050319364e-14
  4.8599800162611e-15
-1.95862184650587e-14
\end{lstlisting}
    The vector X we have found here can be interpreted as a point in a 3D space, where, for example 
\begin{align*}
    x = -7.57416762700087 \\ 
    y = 0.0158780881981559 \\
    z =  5.13453514211295
\end{align*}
    This point could then represent the intersection point of the three planes given in the system of this exercise. \\
    We notice that the values for the residual vector are extremely small, this indicates that the resulting 
    vector X is very precise.

    \subsubsection{Condition number of A}
    To calculate the condition number of A, we will use the following definition: \[\kappa(A)= \frac{|max(S)|}{|min(S)|} \]
    where S is the vector of singular values of A. In GSL, we can find S by using gsl\_linalg\_SV\_decomp.
\begin{lstlisting}
gsl_linalg_SV_decomp(U, V, S, work);

double condNumber, minS, maxS;
minS = gsl_vector_get(S, 0); maxS = gsl_vector_get(S, 0);
for (int j = 0; j < size_j; j++) {
    if (gsl_vector_get(S, j) < minS) minS = gsl_vector_get(S, j);
    if (gsl_vector_get(S, j) > maxS) maxS = gsl_vector_get(S, j);
}
condNumber = fabs(maxS) / fabs(minS);
\end{lstlisting}
    After doing this, we can generate the following output:
\begin{lstlisting}
Vector S, singular values of A, result of doing SV decomposition:
     13.7188626226398
     6.13888725147292
   0.0243650331438208

Condition number = 13.7188626226398 / 0.0243650331438208 = 563.055364696643
\end{lstlisting}
    We get a condition number of about 563. Informally, we know that
    \begin{quotation}
    As a rule of thumb, if the condition number  \(\kappa(A) = 10^k\), 
    then you may lose up to k digits of accuracy on top of what 
    would be lost to the numerical method due to loss of precision 
    from arithmetic methods.\footnote{https://en.wikipedia.org/wiki/Condition\_number}
    \end{quotation}
    This way, we can quickly estimate that with a condition number of 563 we \emph{may} lose approximately 
    3 digits of accuracy (in the vector X) on top of the loss caused by machine precision.\\
    
    Next, we will use the condition number and more formally describe the maximum inaccuracy 
    which \emph{may} occur after these calculations.
    
    \subsubsection{Maximum possible error}
    We know the following inequalitíes which express that the maximum possible relative errors are bounded:
    \[\frac{\mid\mid y - Ax\textsubscript{*} \mid\mid}{\mid\mid A \mid\mid \ \mid\mid x\textsubscript{*}\mid\mid} \leq \rho \epsilon,\ \ \ \ \ 
    \frac{\mid\mid x - x\textsubscript{*} \mid\mid}{\mid\mid x\textsubscript{*} \mid\mid} \leq \rho \kappa(A) \epsilon \]
    Here, \(\epsilon\) is the relative machine precision eps. For a double, 
    this is \(2.22\mbox{\sc{e}-}16 \).\footnote{https://en.wikipedia.org/wiki/Machine\_epsilon\#Values\_for\_standard\_hardware\_floating\_point\_arithmetics}
    If we take the approximately worst case for p, which is around 10, then we get that \(\rho \epsilon = 2.22\mbox{\sc{e}-}15 \)

    \newpage
    In the second formula, we see a better way to describe the impact that the condition number has on the
    accuracy of the calculations for \(x\). For \(\rho \kappa(A) \epsilon\), we find:
\begin{align*}
    \rho \kappa(A) \epsilon \approx 10 * 563.055364696643 *2.22\mbox{\sc{e}-}16\\ 
    \rho \kappa(A) \epsilon \approx 1.24998290962654746\mbox{\sc{e}-}12 \\
    \rho \kappa(A) \epsilon \approx 1.25\mbox{\sc{e}-}12
\end{align*}
     We can now say that our result for \(x\) will have a maximal error of \(1.25\mbox{\sc{e}-}12\),
     with respect to the exact solutions for \(x\).
    
    \subsection{Solving the system after changing a\textsubscript{2,2}}
    In this section, we change a\textsubscript{2,2} from -4.273 to -4.275, and just repeat all of the steps.
    The most important parts of the output are shown below.
\begin{lstlisting}
Vector X, result of solving with LU decomposition:
    -7.57509282124123 
   0.0157630382743384
     5.13497563543488

Vector r, residual of solving by LU decomposition:
 9.58958990610966e-14
 1.12524872037495e-14
-4.64732888184761e-14

Condition number = 13.7190169534295 / 0.024538406356238 = 559.083452864162
\end{lstlisting}
    We notice that the condition number has decreased, which means the maximum inaccuracy that \emph{may}
    occur also decreassed. We see this when calculating \(\rho \kappa(A) \epsilon\):
\begin{align*}
    \rho \kappa(A) \epsilon \approx 10 * 559.083452864162 *2.22\mbox{\sc{e}-}16\\ 
    \rho \kappa(A) \epsilon \approx 1,241165265\mbox{\sc{e}-}12 \\
    \rho \kappa(A) \epsilon \approx 1.24\mbox{\sc{e}-}12
\end{align*}
    Coincidentally, the residual vector \emph{increased} in this case, which means the calculated solution
    for the system of equations became less accurate. \\
    This can be interpreted graphically as the point in 3D space moving away from the intersection of
    the three planes.\\
    
    Thus, we have found that changing a\textsubscript{2,2} from -4.273 to -4.275 has decreased the condition
    number for the matrix, but increased the residual vector. This is perfectly possible, with the condition
    number we can only make a statement about the maximum \emph{possible} error, we cannot say anything about
    the actual error without knowing the exact solution.
\section{Conclusion}
    We have solved the given system of equations and found the following vector as a solution.
\begin{lstlisting}
Vector X, result of solving with LU decomposition:
    -7.57509282124123
   0.0157630382743384
     5.13497563543488
\end{lstlisting}
    The maximum possible error of this vector X, with respect to the exact solutions for \(x\),
    is approximately \(1.25\mbox{\sc{e}-}12\). \\
    Graphically, this can be interpreted to be the point in 3D space, that is the intersection of three planes.\\
    
    We have found that changing a\textsubscript{2,2} from -4.273 to -4.275 decreased the maximum possible error
    to approximately \(1.24\mbox{\sc{e}-}12\). However, we noticed that the actual error of the vector X increased in this case.\\
    Graphically, we can say that the point moved away from the intersection of the planes.

\onecolumn
\appendix
\appendixpage
\addappheadtotoc

\section{Code}
	\subsection{main.cpp}
		\lstinputlisting[basicstyle=\scriptsize]{../main.cpp}
	\bigskip
	\subsection{createImages.sh}
		\lstinputlisting[basicstyle=\scriptsize]{../createImages.sh}

\newpage
\section{Output}
	\subsection{output.txt}
		%\lstinputlisting[basicstyle=\scriptsize]{../build/output.txt} 
		
\end{document}