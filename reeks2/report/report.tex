\documentclass[11pt, a4paper, titlepage, openright]{article}
% \usepackage[options]{package}

\title{\LARGE Scientific Programming \\ \normalsize Data Fitting Exercise 1}
\author{Armin Halilovic - s0122210}
\date{October 30, 2015}

\usepackage[font=small,labelfont=bf]{caption}
\usepackage{float}
%\floatstyle{boxed}
\restylefloat{figure}
\usepackage{graphicx}
\usepackage{hyperref}
\usepackage{mathtools}
%\usepackage{titlesec}
\usepackage[titletoc, title]{appendix}
\usepackage{listings}
\usepackage{color}

\definecolor{dkgreen}{rgb}{0,0.6,0}
\definecolor{gray}{rgb}{0.5,0.5,0.5}
\definecolor{mauve}{rgb}{0.58,0,0.82}

\lstset{frame=tb,
  language=C++,
  aboveskip=3mm,
  belowskip=3mm,
  showstringspaces=false,
  columns=flexible,
  basicstyle={\footnotesize\ttfamily},
  numbers=none,
  numberstyle=\tiny\color{gray},
  keywordstyle=\color{blue},
  commentstyle=\color{dkgreen},
  stringstyle=\color{mauve},
  breaklines=true,
  breakatwhitespace=true,
  tabsize=3,
  showstringspaces=false
}

\begin{document}
\input{title_page.tex}
\onecolumn
\tableofcontents
\newpage
%\twocolumn


\section{Problem}
\section{Using the code}
    All of the C++ code can be found in the "main.cpp" file and in appendix A of this document.
    main.cpp comes accompanied by createImages.sh, which contains all of the neccessary UNIX commands to generate the graph images.
    This file relies on the \href{https://www.gnu.org/software/plotutils/manual/en/html_node/graph.html}{graph} program
    in the GNU plotutils package to plot graphs, so make sure that it is installed.

    To compile and run the program, execute the following commands in the build/ directory:
    \begin{lstlisting}
    cmake ..
    make
    chmod +x ./createImages.sh
    ./data_fitting
    \end{lstlisting}
    All of the graphs should be present in the build/images/ directory. If they are not there, make createImages.sh executable
    with "chmod +x" and run it to create them.

\newpage
\section{Solutions}
\subsection{Polynomial through non equidistant points}
\subsection{Natural cubic spline}
\subsection{Cubic spline with modified conditions}

\section{Conclusion}
    After conducting the interpolations, we can conclude that using a cubic spline is the best way to
    approximate the Runge function, if we had to choose between polynomial and spline interpolation.
    This conforms to what we learned during the lecture on data fitting.



\onecolumn
\appendix
\appendixpage
\addappheadtotoc

\section{main.cpp}
    \lstinputlisting[basicstyle=\scriptsize]{../main.cpp}
    \newpage

%\section{createImages.sh}
%    \lstinputlisting[language=bash, basicstyle=\scriptsize]{../createImages.sh}

\end{document}

/iffalse

\[f(x) = \frac{1}{1 + 25 x^{2}} \ \ x \in [-1, 1] \]
\subsection{Polynomial through equidistant points}
\label{sec:firstpoly}
\begin{lstlisting}
int i = 0, size = 17;
double xi, xa1[size], ya1[size];
for (xi = -1; xi <= 1; xi = xi + ( 2 / (double) 16)) {
    xa1[i] = xi;
    ya1[i] = runge_f(xi);
    i++;
}
\end{lstlisting}
\href{https://www.gnu.org/software/gsl/manual/html_node/1D-Higher_002dlevel-Interface.html#g_t1D-Higher_002dlevel-Interface}{gsl\_spline}
In figure~\ref{fig:runge}, we can see what the Runge function looks like.
\begin{figure}[H]
    \centering
    \includegraphics[width=9cm, trim={2cm, 4cm, 2cm, 3cm}, clip]{../images/spline2}
    \caption{Spline interpolation through equidistant points}
    \label{fig:spline2}
\end{figure}
\begin{figure}[H]
    \begin{minipage}[b]{0.49\textwidth}
        \includegraphics[width=6.5cm, trim={2cm, 4cm, 2cm, 3cm}, clip]{../images/diff4abs}
    \end{minipage}
    \hfill
    \begin{minipage}[b]{0.49\textwidth}
        \includegraphics[width=6.5cm, trim={2cm, 4cm, 2cm, 3cm}, clip]{../images/diff4rel}
    \end{minipage}
    \caption{Differences for spline interpolation}
    \label{fig:diff4}
\end{figure}
\bigskip

/fi